\documentclass{article}

\begin{document}

\section{Vocabularies}

\begin{itemize}
    \item Multispectral volumetric MRI record
    \item Multimodal (MR images)= T1, T2, FLAIR, post-Gadolinium T1.\\
    The two basic types of MRI images are T1-weighted and T2-weighted images.
    \item T1 = T1 images result in images which highlight fat tissues within the body 
    \item T2 = T2 images result in images which highlight fat and water within the body
    \item FLAIR = Fluid attenuated inversion recovery
    \item \textbf{Volume}
    \item \textbf{record}
    \item detection 
    \item segmentation
    \item voxel = cubic pixel, element of volume
    \item tissue = Gewebe
    \item tissue type = tumor, edema, negative
    \item edma = also known as fluid retention or swelling, is the buildup of fluid in the body's tissue.
    \item Glioma = a type of tumor (contains multiple tyes of cancer) that occurs in the brain and spinal cord
    \item Truth image
    \item Intensity at imaging = 8bit grey scale has 256 different kinds of grey. That is the Intensity
    \item absolute intensity = 
    \item slice =
    \item feature = an individual measurable property or characteristic of a phenomenon being observed
    \item label = result, decision
    \item sample = the selection of a subset
    \item entropy based criteria
    \item random forest = ensemble learning method which tackles the overfitting problem of single decision trees by selecting the class of a data set through majority count of many trees (forest) which only select a few features randomly. 
    \item weak learner = a classifier that is only slightly correlated with the true classification (it can label examples better then random guessing)
    \item stron learner = a classifier that is arbitrarily well-correlated with the true classification
    \item overfitting = the production of an analysis that corresponds too closely or exactly to a particular set of data, and may therefore fail to fit additional data or predict future observations reliably
\end{itemize}

\end{document}